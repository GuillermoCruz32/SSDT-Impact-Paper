\section{Related Work}
% We may merge this section with Introduction. It's flowing like a related work already.

The switch from paper to computer is not always easy. (at a liberal arts institution?) There are those that resist these changes and have a negative attitude toward technology. There is evidence suggesting that age differences in informational processing have an effect on older workers' performance of computer-based tasks \cite{oldpeopleandtech}. 
However, an in-house team of student software developers can create user-friendly interfaces. Through in depth* (needs rephrase) communication with customers and usability testing \cite{usabilitytesting}, the team can ensure the implementation is simple and effective for meeting the customer's needs.

(something about professors dealing with moodle's bullshit//similar instances with institutions outsourcing software development)

(Something about the first undergraduate software team here)\cite{rochesterfirstundergradsoftwareteam}

(Something about this proposed framework here. "A Scalable and Portable Structure for Conducting Successful Year-long Undergraduate Software Team Projects") \cite{yearlong}

(Something about how we are unique in that we don't do course work/capstone projects; we make software for the institution.

[CITE THIS https://www.nytimes.com/2014/08/28/nyregion/students-inventing-programs-to-streamline-their-colleges-data.html] 
In 2013, 19 students at Baruch College used a script to query for openings in crowded courses; they did so at such high frequencies that they nearly took down their institution's system. A student at Rutgers University grew frustrated at his inability to get into popular courses; he built a tool that queried the university's registration system and by the next semester, 8,000 people had used it. These encounters show the difficulty of retrieving necessary data from dated school information systems. 
[To be continued...]


(something about scotts paper cause he said so lmao maybe move this to the end of results?? talk about how the millenials are doing in our program?) \cite{heggen2018hiring}