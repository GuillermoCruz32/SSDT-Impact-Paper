\section{Methods}
%Needs paragraph formatting, content chunking, etc
\subsection{The Team}
Student Software Development Team (referenced henceforth as the SSDT) at our institution is comprised of six to ten undergraduate students of all levels who typically are majoring or minoring in Computer and Information Sciences, two staff members and a Computer Science faculty member who serves as the Scrum Master. (Should we include details about the Labor Program?) The SSDT work year-round improving, designing, developing, and managing software for the institution using Agile methodologies.

Typically, a new software development cycle takes place at the beginning of the summer in May with a new group of students. It is most custom to have no more than three students who are returning from a past development cycle. These students are rising sophomores, juniors and seniors. The development cycle is broken up into three segments: Summer term, Fall Term, and Spring term. This follows the traditional structure of the College's academic schedule. The Summer term is structured differently than the Fall and Spring terms as the students are not enrolled in classes while simultaneously being employed in this position. The Summer term operates like an internship and consists of nine weeks where the students work for 40 hours weekly. This allows the students to gain 360 hours of hands-on, technical experience before the Fall term begins. 

The Fall and Spring terms are both comprised 16 weeks* where most students work at least 10 hours a week. Upperclassmen can work 12 to 15 hours a week if they work it out with their supervisor. This change in the number of hours students working weekly occurs because of the start of class during the Fall semester. BC is one of nine [CITATION NEEDED] work colleges in the United States; These colleges offer students a unique value proposition by greatly reduced tuition in exchange for service to the college and surrounding community. This service typically includes incorporating labor into the curriculum. Students work at the school for a set number of hours per week and it is credited to the student in the form of reduced tuition. Students learn real-world work skills such as time management, and responsibility. Some students move up into positions of leadership and build those skills. Between classes, extracurriculars, and secondary labor positions, SSDT's staff dedicate time to their labor hours.

The SSDT receives requests from departments and offices at the college who are in need of improvements to their current software, a complete refactoring of their current software, or in some cases, are still using inefficient paper processes and have no software currently in use. Typically project requests are received via email or during a discussion of the customer's current status of their processes. It is also not unlikely for the Scrum Master to reach out to offices and departments throughout the institution and inquire about the existing software (or lack thereof).  When a software request is received, the team* reviews the components of each request and determine which software to prioritize for the year. The team decides which software request will be the main focus based on the which has the greatest need. This decision is based on the request's urgency, importance, value to the institution, and the effort it will take to implement. Urgency is determined by how soon the requestor needs the software, i.e. if the requestor's office is currently running so inefficiently on their current processes that it affects other offices on campus. Importance and value is considered based upon how largely the implementation will be used at the institution. Last, effort is evaluated by the time, number of student programmers it will take to bring the request to completion. All of these factors are weighed in order to select the request that requires the most immediate attention. After a request is determined* to need improvements in all of these areas, it is selected as the center of focus for the team.

\subsection{The Process}

\subsubsection{An Agile Scrum Approach}
After a software request is selected, the team follows the principles and values described in the Agile Manifesto to begin the planning. The Manifesto does not provide concrete or descriptive instructions on how to develop software but instead provides fundamental information to be considered throughout the entirety of the software development from project initiation to project close. Agile software development prioritizes interactions between the software team as well as the customer and encourages software processes that are receptive to change while delivering quality software. The Scrum methodology is a subgroup of the Agile project management framework that articulates more details and specifications on how to employ the principles in the Manifesto within the team's software development practices.  "the goal of delivering new software capability every 2-4 weeks" [CITATION NEEDED]. Scrum is the most popular agile methodology; "According to the 12th annual State of Agile report, 70 percent of software teams use Scrum or a Scrum hybrid" [CITATION NEEDED]. The hybrid of Scrum used by the SSDT also intertwines the Kanban model in its practices. The Scrum methodology dictates that there are three key roles in which all those involved with the development of a software fall under: Product Owner, Scrum Master, and Development Team. The Product Owner, in relation to the SSDT, is the individual (or group of people) who makes a software request and is essentially a customer who conveys the vision and the mission of the software product to the SSDT. Newly added features, fixed bugs, and completed software must be approved and accepted by the customer before it is labeled as "Done". The Scrum Master, the Computer Science faculty in the case of the SSDT, is charged with the responsibility of facilitating the Development Team and communicating the needs of the Product Owner. In consonance with the Scrum approach, the SSDT does not have a team leader who makes decisions for the whole or decide how problems are solved, but instead has a Scrum Master. The Scrum Master supports and promotes the Development Team and ensures that the SSDT understands and practices Scrum theory. The final categorization of Scrum roles is the Development Team, which is represented by the undergraduate students, according to The Scrum Guide are "structured and empowered by the organization to organize and manage their own work", [CITATION NEEDED] hence the absence of the role of a team leader. Scrum theory provides an iterative, incremental approach to cut down on risks and enhance the predictability of the project. Applying the Kanban model in tandem with the Scrum framework aids in managing the overall flow of the project. The Kanban model prioritizes three essential principles: visualize what will be done today,  reduce the amount of work that is in progress, and properly manage the flow. Kanban encourages continued cooperation and creates an environment that promotes ongoing learning by describing the optimal team workflow.

Scrum prescribes four events in order to carry-out inspection and adaptation. The events consist of the Sprint Planning [TALK ABOUT PAPER PROTOTYPING HERE(I put more info down below. want some more here or nah?-kat], Daily Scrum and Sprint, Sprint Review, and Sprint Retrospective. Scrum complemented with Kanban references these events as "flow-based events" to acknowledge the importance of managing the teams workflow during these events. The Summer term, when students work for 40 hours a week, is when the SSDT can execute these events to the full extent as the team is able to convene in a way that is comparable to that of a software team that works full-time year round. The Sprint Planning begins when the SSDT decides which customer request will be the central focus for that year. This event is used to outline the work that will be performed during the Sprint. The Development Team uses this time to forecast the range of capabilities that will be developed during the Sprint and to also set the Sprint Goal. The Sprint Goal for the SSDT is to have a beta product available by the end of the Sprint.  When a software is released in beta, the majority of the software requirements have been met, however,there may be small issues that have yet to be addressed.  By releasing beta versions of products, the SSDT and the Product Owner are able to observe most of the functionality of the software that has already and also test for inconsistencies. Close communication with the customers is crucial to the development of the software; staff's needs may change, college policies may update, and staff may move on to jobs elsewhere. As needs change, SSDT can adapt to and account for them.

Sprint Planning begins with analyzing the current processes that the Product Owner is utilizing*. Asking questions such as, What does the current software that is used to solve their problem look like? How does it work? Are they using any software? Where is the data that needs to be tracked currently being stored? What forms have to be filled out and which people have the authority to approve this process before it is considered to be done and put into action. For instance, if the Product Owner was requesting a software that allows students to add and remove courses to their schedules. The SSDT would need to know how students are currently able to do this task and then make sure to add it to the requirements of the software that is being built. During the SSDT's most recent Sprint, the request that was chosen involved doing an entire refactoring of a live* software. In order to get a better idea of the functionality of the new software should be implemented, the Development Team begins by going through multiple iterations of paper prototypes. Paper prototyping is a prototyping method in which paper is used to simulate a computer or web application. A paper prototype should hold all of the functionality that the finished user interface of the application will have; from navigation bars, drop down menus, and headers to button clicks and items that will hover on the interface. A person should be able to "click-through" the website via the paper prototype. To start this process, screenshots of the old software's interfaces were taken and printed out. Next these interfaces are critiqued* and analyzed to decide which parts are to be kept for the refactoring, which parts will need to be reworked, and which parts will need to be discarded altogether. The purpose of this part of the paper prototyping process is to become familiar with the current software so that the team can adequately build a better software. The team looks for things* such as bugs, broken links, slow page loading, poor user design, etc. and makes a note of these issue to ensure that they are addressed in the new software. After the individual interfaces of the previous software have been discussed and analyzed, the team divides the different interfaces into two categories: Main and Administrative. Main interfaces are the web pages that all users of the application will be able to see whilst Administrative web pages will only be accessed by system administrators, such as BC staff and SSDT's supervisor. 


Next the team breaks down the interfaces into issues in which the team will tackle* in pairs. The pair of developers then begins to design the interface that they had chosen from the interfaces that needed to be refactored. Each time a pair from the team believes that they have successfully created a valid* paper prototype, it is then tested for usability. Paper prototyping testing occurs* in  the same way that a fully functional software would be tested, the person testing the paper interface will treat it as such. The tester will "click" on different components of the interface and the designers will physically move parts and replace parts of the paper interface in order to replicate how the real software would respond. The pair who designed the interface will take notes as the tester navigates their paper web page and when the test is complete, they use these notes to redesign and the process is repeated. Paper prototyping is done because "bugs" and inefficiencies are easier to fix when no coding has been done yet and one can go through many iterations without having to actually having to troubleshoot actual code. The SSDT repeats this process of designing, testing, and re-designing until the entire team is satisfied with the final iteration. After all interfaces have been drafted and prototyped, the process of building the application begins and the team commences the Sprint.


\subsubsection{[ INCLUDE DDS SOMEWHERE IN HERE] DAILY SCRUM & SPRINT}
The SSDT uses the Flask framework to build most of its software, following the MVC architectural model. Flask is a web framework that uses Python and Jinja; it is designed to make the start up of web applications faster and easier. It also allows developers to decide what tools and libraries they would like to use for their web application. MVC is architectural model that separates an application into three main logical components: the Model, the View, and the Controller. The Model represent the data and the logic that the user works with in the application. The View is the is composed of all of the User Interface (UI) parts of the application meaning the parts of the application that you can physically see. The Controller is the bridge between the Model and the View; it processes all of the logic, handles requests, manipulates data, and renders the output of the View. All but two applications built by the SSDT were implemented using Flask MVC features. The Sprint is set for 6 weeks with the goal of completing many of the main interfaces in order to have a demo available for  the Product Owner to preview and test.

During the Sprint, the team convenes everyday in the morning in order to touch base with each of the pairs within the team and get an idea of their progress. During the Scrum meeting, the smaller teams tell what the have accomplished since the last meeting, what they will begin to work on next and what they will be completing before the next Scrum meeting, and where they are stuck or if any obstacles got in the way of completing their previous work. Scrum meeting (also called the Daily Stand-Up) is for quick communications purposes and should take no longer than 15 minutes. Sometimes small demos are done during the Scrum meeting in order to get the entire teams input on certain features before proceeding. During the Fall and Spring terms when the SSDT's schedules are less coherent* than the Summer, the programmers are expected to do a "DDS" in a team collaboration application called Slack, DDS stands for Did, Doing, and Stuck. This update stands in place* of the daily Scrum meeting all students are not available at the same time daily. Team leaders examine the DDS reports to ensure progress is steady as well as evaluate who needs guidance. These updates also aid in communication between team members if they happen to not see each other. Instead of having a daily Scrum, we instead have a weekly Labor Meeting in which we evaluate the team's progress and state in production.

\subsubsection{SPRINT REVIEW}
After the Sprint is completed, the team does a larger demonstration to present the work that they have completed during the Sprint. After the team demonstration, the Product Owner also participates in the Review; their job is to make sure the work so far meets their criteria and requirements. They also have the authority to reject part of the application, suggest modifications to what has been built, or request a feature be added to the application. The feedback from the Product Owner during this stage is essential as it defines new criteria for the next wave of implementation.

\subsubsection{SPRINT RETROSPECTIVE}
This is the final meeting after the Sprint in which the team discusses the overall process and performance of the Sprint. This is an opportunity for the SSDT and the Scrum Master to strategize ways to improve their methods and approaches for the next Sprint. The team's strengths and triumphs are examined alongside the setbacks to better prepare for these challenges in the future.
