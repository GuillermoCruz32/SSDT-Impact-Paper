\section{Methods}
%%This section is due Sept 14th!!
%Needs paragraph formatting, content chunking, etc
A.   The Team
The Berea College Student Software Development Team (referenced henceforth as the SSDT) is comprised of six to ten undergraduate students of all levels who typically are majoring or minoring in Computer and Information Sciences, two staff members (one of whom is an alumni of the Computer Science program), and a Computer Science faculty member who serves as the Scrum Master. (Should we include details about the Labor Program?) The SSDT work year-round improving, designing, developing, and managing software for the institution using Agile methodologies.

Typically, a new software development cycle takes place at the beginning of the summer in May with a new group of students. It is most custom to have no more than three students who are returning from a past development cycle. The development cycle is broken up into three segments: Summer term, Fall Term, and Spring term. This follows the traditional structure of the College's academic schedule. The Summer term is structured differently than the Fall and Spring terms as the students are not enrolled in classes while simultaneously being employed in this position. The Summer term consists of nine weeks where the students work for 40 hours a week. This allows the students to gain 360 hours of hands-on, technical experience before the Fall term begins The Fall and Spring terms are both comprised 16 weeks* where most students work at least 10 hours a week. Upperclassmen can work up to 15 hours a week. The change in the number of hours students work per week occurs because of the start of class during the Fall semester.

The SSDT recieves requests from departments and offices at the college who are in need of improvements to their current software, a complete refactoring of their current software, or in some cases, are still using inefficient paper processes and have no software currently in use. Typically project requests are recieved via email or during a discussion of the customer's current status of their processes. It is also not unlikely for the Scrum Master to reach out to offices and departments throughout the institution and inquire about the existing software (or lack thereof).  When a software request is recieved, the team* reviews the components of each request and determine which software to prioritize for the year. The team decides which software request will be the main focus based on the which has the greatest need. This decision is based on the request's urgency, importance, value to the institution, and the effort it will take to implement. Urgency is determined by how soon the requestor needs the software, i.e. if the requestor's office is currently running so inefficiently on their current processes that it affects other offices on campus. Importance and value is considered based upon how largely the implementation will be used at the institution. Last, effort is evaluated by the time, number of student programmers it will take to bring the request to completion. All of these factors are weighed in order to select the request that requires the most immediate attention. After a request is determined* to need improvements in all of these areas, it is selected as the center of focus for the year.

B. The Process

I. An Agile Scrum Approach
After a software request is selected, the team follows the principles and values described in the Agile Manifesto to begin the planning. The Manifesto does not provide concrete or descriptive instructions on how to develop software but instead provides fundamental information to be considered throughout the entirety of the software development from project initiation to project close. Agile software development prioritizes interactions between the software team as well as the customer and encourages software processes that are receptive to change while delivering quality software. The Scrum methodology is a subgroup of the Agile project management framework that articulates more details and specifications on how to employ the principles in the Manifesto within the team's software development practices.  "the goal of delivering new software capability every 2-4 weeks" [CITATION NEEDED]. Scrum is the most popular agile methodology; "According to the 12th annual State of Agile report, 70% of software teams use Scrum or a Scrum hybrid" [CITATION NEEDED]. The hybrid of Scrum used by the SSDT also intertwines the Kanban model in its practices. The Scrum methodology dictactes that there are three key roles in which all those involved with the development of a software fall under: Product Owner, Scrum Master, and Development Team. The Product Owner, in relation to the SSDT, is the individual (or group of people) who makes a software request and is essentially a customer who conveys the vision and the mission of the software product to the SSDT. Newly added features, fixed bugs, and completed software must be approved and accepted by the customer before it is labeled as "Done". The Scrum Master, the Computer Science faculty in the case of the SSDT,  is charged with the responsibility of facilitating the Development Team and communicating the needs of the Product Owner. In consonance with the  Scrum approach, the SSDT does not have a team leader who makes decisions for the whole or decide how problems are solved, but instead has a Scrum Master. The Scrum Master supports and promotes the Development Team and ensures that the SSDT understands and practices Scrum theory. The final categorization of Scrum roles is the Development Team, which is represented by the undergraduate students, according to The Scrum Guide are "structured and empowered by the organization to organize and manage their own work", [CITATION NEEDED] hence not needing a team leader. Scrum theory provides an iterative, incremental approach to cut down on risks and enhance the predictability of the project. Applying the Kanban model in tandem with the Scrum framework aids in managing the overall flow of the project. The Kanban model prioritizes three essential principles: visualize what will be done today,  reduce the amount of work that is in progress, and properly manage the flow. Kanban encourages continued cooperation and creates an environment that promotes ongoing learning by describing the optimal team workflow.

Scrum prescribes four events in order to carry-out inspection and adaptation. The events consist of the Sprint Planning [TALK ABOUT PAPER PROTOTYPING HERE], Daily Scrum and Sprint, Sprint Review, and Sprint Retrospective. Scrum complemented with Kanban references these events as "flow-based events" to acknowledge the importance of managing the teams workflow during these events. The Summer term, when students work for 40 hours a week, is when the SSDT can execute these events to the full extent as the team is able to convene in a way that is comparable to that of a software team that works full-time year round. The Sprint Planning begins when the SSDT decides which customer request will be the central focus for that year. This event is used to outline the work that will be performed during the Sprint. The Development Team uses this time to forecast the range of capabilities that will be developed during the Sprint and to also set the Sprint Goal. The Sprint Goal is generally to have a Beta product released with the majority of the software requirements met even if there are small issues that are yet to be addressed.

hi

 Measuring Success
The objective of the study is to demonstrate the impact of software developed predominantly by undergraduate students has on higher educational institutions. To determine the impact that the student-developed software has had on the institution several factors of software success are identified and evaluated.  There exists much research on quantifying project success. For this study, the metrics for software success were defined using DeLone and McLean’s Model for Information Software Success. (CITATION NEEDED) This model consists of six measures of success which included the following: system quality, information quality, use, user satisfaction, individual impact, and organizational impact. These measures may be used to evaluate the software systems based on factors including but not limited to, code quality, appearance, usability, relevance, and cost reduction.  A survey comprised of questions that derive from each of these six components was constructed. There are seven software systems that are currently in production at the institution allowing for a multiple case study (CITATION NEEDED).

    Participants
Participants in this study included [INSERT NUMBER of PEOPLE HERE] Berea College faculty and staff. All participants in this study were volunteers and were administered a survey pertaining only to the software that they had interacted with. The survey was only sent to those who were the primary users and/or had administrative privileges to the software. This pool of faculty members was selected, as opposed to surveying all users, because these subjects were involved in the initial phase of the software development process, which is defining the software project’s scope and the needs of the customer. [CITATION?] Each survey began with an informed consent disclaimer that detailed the nature

The SSDT has eight systems that we handle. These include the Advancement Office, BCAC, BCSR, CAS, LSF, Ulmann, URCPP, and SKYZ. Each system has a specific functionality to aid our faculty and staff clientele perform their jobs. For example, CAS (Course Administration and Scheduling) helps the registrar, as well as faculty and staff, schedule terms, manage courses, programs, divisions, and much more.
We distributed our surveys and gave our clients two weeks to submit their responses. We barely got over 50 percent participation (9/17 survey responses were submitted)


%%should we have a paragraph for each system? Acronym breakdown, client, and system description??  Kat
Quantitative data (stats)
%I think we should include all possible responses for questions like "how important is it" to show what responses the clients were given to choose from
Of our 9 responses: %I started at column D in Data Analysis spreadsheet since the first three were not quantitative
How often do you use the software? daily use: 3, weekly use: 1, monthly use: 4, yearly use: 1.
Each time you use the  application, how long do you interact with it?: < 15 minutes: 1, 15-30 mins: 3, 30 mins - 1 hr: 4, 1 - 2 hrs: 1.
How much of your day-to-day responsibilities depend on your ability to use the  application? Important: 4, Essential: 5.
How much do you agree with each of the following statements?
[Using this application allows me to do my job better]: Strongly Disagree: 1, Neutral: 1,Agree: 1, Strongly Agree: 6.
[Using this application allows me to do my job faster]: Strongly Disagree: 1, Disagree: 1, Agree: 1, Strongly Agree: 6
[My level of competency with software is not relevant when using the software] Disagree: 1, Neutral: 4, Agree: 2 Strongly Agree: 2
%The next questions are system specific functionality/ease of use so we will have to represent that data in some other way...Since each system had different (sometimes 1, sometimes multiple, functionalites here (cas has a lot, BCSR has 1.)
%If system was gone, what would it cost client?
If the software were to be taken down entirely, what would it cost you in terms of:
Stress:
Time:
Errors:
Money:
%Comparison of system to other non SSDT software(moodle,banner, etc)
How would you estimate the value of the software compared to Berea College's following applications?
Banner:
MyBerea:
Degree Works:
Moodle:
Online Bookstore:
TRACY:
College email:
Password Management site:
Hutchins library site:
