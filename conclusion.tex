\section{Conclusion}
In conclusion, higher educational institutions, like most all large businesses, require software to manage their daily operations. Because of limited budgets, the costs to purchase some software is prohibitive, resulting in the purchase of software that is poorly developed, unsupported, or doesn't fit the customer's needs. Worse yet, contracts lock institutions into this software for multiple years, putting a strain on the faculty, staff, and students who rely on the software, despite it not adequately supporting their work. As an alternative solution, a team of student developers supported by a faculty and staff can create custom, institution-specific software that meets the college's unique needs. By following industry best-practices, such as lean thinking and agile methodology, students can create quality software. As an added bonus, students gain valuable career experience building their technical and soft skills, preparing them for software engineering positions after graduation. 

Detractors may indicate that trusting undergraduates with this task, even under the leadership of faculty or staff, is an unacceptable risk to the institution. However, the Student Software Development Team presented here counters that argument, proving that a program can exist without putting the institution at risk through the careful selection of projects and a framework for guiding and monitoring the students as they create code. Product owners for six of the eight existing applications developed by the SSDT have evaluated their satisfaction with the software, the individual impact of the software on their departments, and the organizational impact of the software, revealing that the software is important and valuable across multiple metrics. The applications have quickly became an integral part of the business of the institution, improving the businesses processes of departments across the institution.