\section{Introduction}
Kats doing this section
[This is the gist. I'm used to writing essays and really short papers, so this needs to be fluffed and fluffed and fluffed some more. Each point should probably be a paragraph.-Past Kat]

As technology advances, users' needs follow suit. Higher educational institutions are forced to adapt with the times, and their processes are making the transition from paper to digital. Many expensive software programs have been developed for these purposes and are on the market. 

However, not all purchasable third-party software is suitable for a college's specific needs. An institution can spend upwards of around 4,000 dollars [I did a simple google search of school administration costs but like..is there something we can cite for this? also i cant put a dollar sign..-kat] on software These applications are built to meet general needs of managing higher education, such as registration, course scheduling, and plagiarism detection \cite{plagarismsoftware}. While it may be a good fit for some functionality, interfaces are likely crafted for a tech-savvy user pool, rendering it difficult to use for older staff members. [Idk this is a bold claim should probably cite it] However, there are processes that take place within an institution's administration that do not fall under the purview of third party software.

An institution's unique needs can be met by a team of students crafting software in-house. Through close communication with customers, the team can create simple, easy to use interfaces that save not only resources by rendering dated paper process unnecessary, but time and money as processes do not take as ling and human error decreases. The students are gaining valuable experience in this setting, preparing them for the workforce after graduation. Along with hard skills gained, such as specific programming languages and Computer Science concepts and practices, students also gain soft skills \cite{softskills} such as leadership, project and time management, and teamwork.

By recruiting a team of student software developers, a college can not only avoid the high cost of software and get applications that are tailored specifically to meet their needs, but also provide students with valuable experience that better prepares them for the industry. 

This paper describes a program which blends the best parts of a software engineering course and an external experience like internships. Through Berea College's work-study program, students are employed by the computer science department to develop software for a full year. Known as the Student Software Development Team, students start in the summer, working approximately 400 hours under the supervision of a faculty member, where they are trained on the software engineering principles and apply them to in-progress projects or new ones. The projects are proposed by the campus community (a.k.a. the customer) and are then selected by the faculty member supervising the team. These projects are often tools requested by the customer to help them complete their daily work, and thus, the software becomes an integral part of the customer’s job. Students intend to finish the beta version of the product by the end of summer and have customer demos and reflection on the team's performance. During the academic year, Fall and Spring terms, the students work an additional 300 or more hours, where they maintain the software; implementing new feature requests, fixing bugs,and performing customer support.

The remainder of this paper is organized as follows: the Related Work section describes research and programs related to undergraduate software engineering. Next,the methodology for implementing meaningful a year-long internship experience is proposed. Finally, we will observe and analyze the specific results of the Student Software Development Team's look into the effectiveness of our software.

