\section{Introduction}

The growing complexity of higher education institutions means an increasing reliance on technology, in particular, software. Most institutions rely heavily on databases and management platforms, such as Banner, to manage the complexity. Similarly, learning management systems such as Moodle improve the educational experience of students and faculty by facilitating and automating common tasks like grading and distributing information to classes. Some of the software is open source, which means the software itself is free, but the customization and support of the software costs the institution. Proprietary software sometimes has an initial purchase cost, a maintenance cost, plus a multi-year contract. While the software can receive good support from the developers and institution-specific customization, the annual costs can creep into thousands of dollars per application. A 2015 report estimated that higher education institutions spend on average 4.2\% of their entire budget on information technology alone \cite{CDSBenchmarkReport}. 

Purchased software often meet the needs of managing higher education, such as course registration and scheduling, but doesn't handle the nuanced differences that exist at each institution. For example, institutions may operate on a quarterly calendar, have online courses without specific start and end dates, or don't operate on a calendar at all. These institution-specific modifications are supported by software developers through ``additional services'' they provide, which increases costs. Sometimes the software simply doesn't exist. For example, work colleges \cite{WCCMembers} require every student to be employed while attending. Since there are only nine, there's little market for software specific to them, yet the need for software still exists. Existing software sometimes solve part of the problem, but often are missing institution-specific features. Again, custom software and modifications to existing software is expensive and adds to the cost of managing the institution. In the case of work colleges, all of which are smaller institutions, this cost is particularly prohibitive. 

An institution's unique needs can be met by a team of students crafting custom, institution-specific software, particularly in the work colleges scenario, where students must have a job. Faculty or staff can serve as a project manager to a team of students, guiding them in the development of software, and the institution can avoid the high cost of custom software. 

In our search for similar programs, there are a plethora of examples of students creating real-world software (i.e., software that is eventually used by the product owner to do their business) in software engineering courses \cite{coursevsproject}, capstone experiences \cite{capstone}, and internships \cite{rochesterfirstundergradsoftwareteam}. However, these examples fail to provide continued support of the software by the students after delivery to the customer, resulting in students missing a major portion of the software Engineering learning (i.e., fixing problems they created and supporting customers). Kaminar \cite{kaminer_2014} summarizes a few instances where students developed software that was adopted by the academic institution. These adoptions occurred organically, and were typically one-off ventures. To our knowledge, no examples were found of institutions hiring students to develop custom software solutions in a multi-semester, internship model, and supporting that software for the entirety of its use at the institution, making this program the first of its kind. 

Another key benefit is that students gain valuable software engineering experience to prepare them for software engineering careers. The students learn numerous technical skills \cite{hardskills}, such as multiple programming languages and software engineering principles. Students also learn valuable soft skills, such as project and time management, teamwork, and problem solving; skills deemed valuable for new graduates \cite{lavy2013soft}. Previous work showed that these benefits are obtained through the development of real-world applications, regardless of the setting (course, capstone, internship, etc.) \cite{liu2005enriching}. The impact of this program on undergraduates was evaluated in previous work \cite{heggen2018hiring}, and thus will not be discussed in detail in this paper. Instead, this paper outlines a program which blends the best parts of a software engineering course, capstones, and internships. In evaluating the program, a survey was issued to product owners, asking them to evaluate the quality and usefulness of the software developed by students. 

% CUT 4p Through a work-study program, students are employed by the Computer Science department to develop software for the institution. Teams of students start in the summer, working approximately 400 hours under the supervision of a faculty member, where they are trained on the software engineering principles and apply them to in-progress projects or new ones. The projects are proposed by the campus community (a.k.a. the product owner) and are then selected by the team. These projects are often tools requested by the product owner to help them complete their daily work, and thus, the software becomes an integral part of their job. Students intend to finish the beta version of the product by the end of summer. During the subsequent academic year (i.e., Fall and Spring terms), the students work an additional 300 or more hours, where they maintain the software, implementing new feature requests, fixing bugs, and performing customer support for their applications. 

% CUT 4p The remainder of this paper is organized as follows: Section 2 describes the framework for employing students as software developers for an academic institution; Section 3 describes the software, how it is selected, and provides examples of software already created; Section 4 presents results from surveys issued to product owners about the effectiveness of the solutions developed by the students; and Section 5 concludes the paper with closing remarks. 
