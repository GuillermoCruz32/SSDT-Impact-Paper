\section{Introduction}
% [This is the gist. I'm used to writing essays and really short papers, so this needs to be fluffed and fluffed and fluffed some more. Each point should probably be a paragraph.-Past Kat]

The growing complexity of higher education institutions means an increasing reliance on technology, in particular, software. Most institutions rely heavily on databases and management platforms, such as Banner \cite{BannerWebsite}, to manage the complexity. Similarly, learning management systems such as Moodle \cite{MoodleWebsite} and Blackboard \cite{BlackboardWebsite} improve the educational experience of students and faculty by facilitating and automating common tasks such as grading and distributing information to classes. Some of the software is open source (e.g., Moodle), which means the software itself is free, but the customization and support of the software costs the institution. Proprietary software typically has a combination of an initial purchase cost, a maintenance cost, and a multi-year contract. While the software can receive good support from the developers and institution-specific customization as-needed, the annual costs can creep into the thousands and millions of dollars. According to a 2015 report about higher education spending, it is estimated that higher education institutions spend on average 4.2\% of their entire budget on information technology alone \cite{CDSBenchmarkReport}. 

%CUT - As technology advances, users' needs follow suit. Higher educational institutions are forced to adapt with the times, and their processes are making the transition from paper to digital. Many expensive software programs have been developed for these purposes and are on the market. 

%CUT - However, not all purchasable third-party software is suitable for a college's specific needs. An institution can spend upwards of around \$4,000 dollars [I did a simple google search of school administration costs but like..is there something we can cite for this? also i cant put a dollar sign..-kat] on software 

These applications are built to meet the general needs of managing higher education, such as course registration and scheduling, but do not handle the nuanced difference that exist at every institution. For example, some institutions operate on a typical semester calendar consisting of a Fall, Spring, and Summer term. Other institutions may operate on a quarterly calendar, or they may have other terms such as a January short-term, multiple summer terms, online courses without specific start and end dates, or not operate on a calendar system at all. These institution-specific modifications are supported by software developers through ``additional services'' they provide. In other words, they are going to cost the institution on top of the base cost of the software.

Worse yet, each institution is unique in its needs, and sometimes the software simply doesn't exist. For example, there are nine work colleges in the United States \cite{WCCMembers, Ecclesia}. A work college requires every student to be employed while attending, and student employment is considered a part of the institution's academic mission. Given there are only nine work colleges, there isn't much market for the creation of large-scale software to support work colleges. However, there is a lot of need for software to manage a work college, such as tracking student employment, time-sheets for clocking hours, reporting back to the federal government and other funding sources, and evaluating student and\/or supervisor performance, to name a few. %Again, custom software adds to the cost of managing the institution...

% Could pay software engineers in-house. Expensive.

An institution's unique needs can be met by a team of students crafting software in-house. Through close communication with customers, the team can create simple, easy to use interfaces that save not only resources by rendering dated paper processes unnecessary, but time and money as these processes do not take as long and human error decreases. 

The students are involved also gain valuable experience in this setting, preparing them for the workforce after graduation. In addition to  the various hard skills \cite{hardskills} that are gained, such as specific programming languages, Computer Science concepts and practices, along with critical thinking and problem solving, students also gain soft skills \cite{softskills} such as leadership, project and time management, and teamwork.

By recruiting a team of student software developers, a college can not only avoid the high cost of software and get applications that are tailored specifically to meet their needs, but also provide students with valuable experience that better prepares them for the industry. 

This paper describes a program which blends the best parts of a software engineering course and an external experience like internships. Through a work-study program, students are employed by the computer science department to develop software for a full year. Teams of students start in the summer, working approximately 400 hours under the supervision of a faculty member, where they are trained on the software engineering principles and apply them to in-progress projects or new ones. The projects are proposed by the campus community (a.k.a. the customer) and are then selected by the faculty member supervising the team. These projects are often tools requested by the customer to help them complete their daily work, and thus, the software becomes an integral part of the customer's job. Students intend to finish the beta version of the product by the end of summer and have customer demos and reflection on the team's performance. During the academic year, Fall and Spring terms, the students work an additional 300 or more hours, where they maintain the software; implementing new feature requests, fixing bugs,and performing customer support. 

The remainder of this paper is organized as follows: the Related Work section describes research and programs related to undergraduate software engineering. Next,the methodology for implementing meaningful a year-long internship experience is proposed. Finally, we will observe and analyze the specific results of the Student Software Development Team's look into the effectiveness of our software.

