
\section{RESULTS}
\subsection{Measuring Success}

The objective of the study is to demonstrate the impact of software developed predominantly by undergraduate students has on higher educational institutions. To determine the impact that the student-developed software has had on the institution several factors of software success are identified and evaluated.  There exists much research on quantifying project success. For this study, the metrics for software success were defined using DeLone and McLean’s Model for Information Software Success. (CITATION NEEDED) This model consists of six measures of success which included the following: system quality, information quality, use, user satisfaction, individual impact, and organizational impact. These measures may be used to evaluate the software systems based on factors including but not limited to, code quality, appearance, usability, relevance, and cost reduction. The software developed by the SSDT uses ideas laid out* in the Agile Manifesto. A survey comprised of questions that derive from each of these six components as well as principles from the Manifesto was sent out to Product Owners of software that was built by the SSDT.

As previously stated, DeLone's and McLean's Model for Software Success were used to model the questions within the survey. Particulary the last four measures were focused on for the survey as the first two (system quality and information quality) are two factors that the Product Owners will not have much interaction with and is more technical. These four factors* are user satisfaction, use, individual impact,and organizational impact. User satisfaction takes into consideration the ease of usefulness of the software. The questions in the survey asked about the simplicity of completing a specific task pertaining to the individual software that Product Owner would have interacted with as well as how learnable the software is. The Product Owners were also asked to rate how useful the features in the software were to them on a scale of one to four (four being the highest). Use of a software factors in the freqency of use, the length of time using the software, and the number of users who actually use* the software. For this component, the users were asked: Does using the application make your job better?,  Does using the application make your job faster?, Do you feel confident while using the application?, and last, Is you level of software competency a relevant factor when using the application? These questions required the Product Owners answer by rating their experiences on a scale of one to five (five being the highest).


B. Evaluating the Software
 The first measure of success is the system quality which ca be defined by the code quality and quality of the system's documentation. The code quality can be measured by its conciseness, testabilty, usability, efficincy, structure, and clarity. This components is primarily
\subsection{Participants}
Participants in this study included [INSERT NUMBER of PEOPLE HERE] Berea College faculty and staff. All participants in this study were volunteers and were administered a survey pertaining only to the software that they had interacted with. The survey was only sent to those who were the primary users and/or had administrative privileges to the software. This pool of faculty members was selected, as opposed to surveying all users, because these subjects were involved in the initial phase of the software development process, which is defining the software project’s scope and the needs of the customer. [CITATION?] Each survey began with an informed consent disclaimer that detailed the nature

The SSDT has eight systems that we handle. These include the Advancement Office, BCAC, BCSR, CAS, LSF, Ulmann, URCPP, and SKYZ. Each system has a specific functionality to aid our faculty and staff clientele perform their jobs. For example, CAS (Course Administration and Scheduling) helps the registrar, as well as faculty and staff, schedule terms, manage courses, programs, divisions, and much more.
We distributed our surveys and gave our clients two weeks to submit their responses. We barely got over 50 percent participation (9/17 survey responses were submitted)


%%should we have a paragraph for each system? Acronym breakdown, client, and system description??  Kat
\subsection{Quantitative data (stats)}
%I think we should include all possible responses for questions like "how important is it" to show what responses the clients were given to choose from
Of our 9 responses: %I started at column D in Data Analysis spreadsheet since the first three were not quantitative
How often do you use the software? daily use: 3, weekly use: 1, monthly use: 4, yearly use: 1.
Each time you use the  application, how long do you interact with it?: < 15 minutes: 1, 15-30 mins: 3, 30 mins - 1 hr: 4, 1 - 2 hrs: 1.
How much of your day-to-day responsibilities depend on your ability to use the  application? Important: 4, Essential: 5.
How much do you agree with each of the following statements?
[Using this application allows me to do my job better]: Strongly Disagree: 1, Neutral: 1,Agree: 1, Strongly Agree: 6.
[Using this application allows me to do my job faster]: Strongly Disagree: 1, Disagree: 1, Agree: 1, Strongly Agree: 6
[My level of competency with software is not relevant when using the software] Disagree: 1, Neutral: 4, Agree: 2 Strongly Agree: 2
%The next questions are system specific functionality/ease of use so we will have to represent that data in some other way...Since each system had different (sometimes 1, sometimes multiple, functionalites here (cas has a lot, BCSR has 1.)
%If system was gone, what would it cost client?
If the software were to be taken down entirely, what would it cost you in terms of:
Stress:
Time:
Errors:
Money:
%Comparison of system to other non SSDT software(moodle,banner, etc)
How would you estimate the value of the software compared to Berea College's following applications?
Banner:
MyBerea:
Degree Works:
Moodle:
Online Bookstore:
TRACY:
College email:
Password Management site:
Hutchins library site:
