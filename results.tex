\section{RESULTS}
% MOVE This internship model has produced such favorable outcomes that it has been implemented consecutively for five years. Early on, the team took on every project that was pitched to the team, resulting in five applications in the first five years. However, by the third year, the team had to bear in mind the maintenance of existing systems in addition to creating new software. Adoption slowed, but did not stop...

\subsection{Study Design}
In order to demonstrate the impact of software developed predominantly by undergraduate students has on higher educational institutions, several factors of software success were identified and evaluated.  The metrics for software success were informed primarily by DeLone and McLean's Model for Information Software Success \cite{delone1992softwaresuccess}. Their model consists of six measures of success: system quality, information quality, use, user satisfaction, individual impact, and organizational impact. Additionally, the SSDT leveraged the Agile Manifesto \cite{agilemanifesto} as a guiding set of principles for developing software. A survey was developed and issued to the product owners of each of the eight applications built by the SSDT. Questions were derived from four of these six components from DeLone and McLean (system quality and information quality were excluded, since product owners are not always able to accurately judge these metrics), as well as principles from the Agile Manifesto. % TODO what metrics from AM and how were our PO's able to judge them?

\subsection{Participants}
Participants in this study included 14 institutional faculty and staff who all represented product owners, in that they were administrators in the systems and had the largest amount of interactions with the software. Ten out of the fourteen clients responded to the survey. The ten respondents represented product owners from six of the eight applications.

\subsection{Evaluation of Software Quality}
\subsubsection{User satisfaction}
To measure user satisfaction, the product owners were asked a set of questions about how easy it was to use and learn the software. The questions asked were: how difficult was it to complete specific tasks in the software; how long did it take to learn the software; how useful were certain features in the software; did the application make their job better; and did they feel more confident while using the application. Table \ref{table:usersatisfaction} summarizes the product owner responses.

\begin{table}
\caption{User Satisfaction - Survey Results. The scale is: Strongly agree (Str. Agr.), Agree (Agr.), Neutral (N.), Disagree (Dis.), and Strongly Disagree (Str. Dis.)}
\label{table:usersatisfaction}
\begin{tabular}{p{2.6cm}p{.75cm}p{.75cm}p{.75cm}p{.75cm}p{.75cm}}
Question: & Str. Agr. & Agr. & N. & Dis. & Str. Dis. \\
 \hline
Software easy to use/learnable & 80\% & 20\% & 0\% & 0\% & 0\% \\
Makes job better & 70\% & 20\% & 10\% & 0\% & 0\% \\
I feel confident using software & 60\% & 40\% & 0\% & 0\% & 0\% \\
Software useful to my job & 90\% & 10\% & 0\% & 0\% & 0\% \\
\end{tabular}
\end{table}

In all, there were 16 questions related to user satisfaction. Respondents were overwhelmingly positive about user satisfaction; 80\% of users indicated the features were ``Very easy to use'' and two answered that the features were ``Slight easy to use''. Only one feature from one application was marked as ``Difficult to use" by one survey respondent.

The users were asked to rate how useful the primary features of the software were to them and to their daily tasks. This is an important measure because when designing software for a customer, it is important that {insert more words here}. The results of this component of the survey showed that nine out of ten respondents answered that the primary feature within the software that they use is ``Essential''; the tenth respondent marked that the primary feature was ``Useful". Moreover, pertaining to all 16 features in the applications, 15 out of the 16 total features were marked as ``Essential'' and one being marked as ``Useful''. This response shows that all of the applications that were built by the student development team were applicable to the responsibilities attributed to them on a daily basis. [Talk about why having useful software makes it successful].

Last in regards to the user satisfaction measurement of software success, the product owners expresses the extent of their agreeance with the following two statements as it pertains to the particular application that they utilize: ``The application makes my job better'' and ``I feel confident while using the application.'' Out of the ten respondents, eight product owners said that they ``Strongly Agree'' that the application makes their job better. [Include why having an application improve quality of work is good] One of the two other users answered that they ``Agree'' with the statement and the last remained ``Neutral'' on the question. In response to their confidence about using the application, seven out of ten of the users answered that they ``Strong Agree" with the statement whilst the other three all answered that they ``Agree'' with the statement. [Talk about why confide is good here]

 \subsubsection{Use}
Use of a software weighs the frequency of use, the length of time using the software, and the number of users who actually utilize the software. For this component, the users were given two statements and were asked to rate each statement based on the degree of  their concordance with the statement on a scale of one to five (five being that they completely agreed with the statement and one being that they did not agree with the statement at all.) The statements included the following: ``The application makes my job faster'' and ``My level of competency with software is not relevant when using the application''. The survey also posed questions about how often each user engaged with the application (daily, weekly, monthly, yearly, once per semester, or never)  and how long the user interacts with the application (less than 15 minutes, between 15 and 30 minutes, between 30 minutes and an hour, between one to two hours, and over two hours per session).

\begin{table}
\caption{Use - Survey Results. The scale is: Strongly agree (Str. Agr.), Agree (Agr.), Neutral (N.), Disagree (Dis.), and Strongly Disagree (Str. Dis.)}
\label{table:usersatisfaction}
\begin{tabular}{p{2.6cm}p{.75cm}p{.75cm}p{.75cm}p{.75cm}p{.75cm}}
Question: & Str. Agr. & Agr. & N. & Dis. & Str. Dis. \\
 \hline
Allows me to do my job faster & 80\% & 10\% & 0\% & 0\% & 10\% \\
My level of software competency irrelevant & 20\% & 20\% & 50\% & 10\% & 0\% \\
\end{tabular}
\end{table}

The use of an application is important because '[describe why this is important]'. When asked if the application allows the user to do their job faster, eight of the user responded that they ``Strongly Agree'' with the statement, one user responded that they ``Agree'' with the statement, and one user remained ``Neutral'' on the statement.

The users were asked if their level of software competency was relevant when using the application. In other words, does the user think that they have to have some level technical knowledge in order to use the application. The responses to this question were more varied than previous questions. Two respondents answered that they ``Strongly Agree'' with the statement,  two responded that they ``Agree'' with the statement, five respondents remained ``Neutral'' on the question, and one respondent answered that they ``Disagree'' with the statement.

The users were also asked how often that they interacted with the software while using it. The answers to the qustion regarding the amount of time that the respondents used the software vary because though all of the users are product owners of one of the application, they have different roles at the institution and utiliize the software differently. This question was followed by a an optional short answer box in case the respondent wanted to elaborate on their answer about usage. It was found that 50\% of the users engaged with the application monthly, one user engaged the software on a weekly and three users engage the software on a dail basis. One user responded that they engaged with the software once per year, however, this result was expected because this software (Undergraduate Research and Creative Projects Program) is used for students at the institution who participate in summer research, thus the application only needing to be used during the summertime.

Last, the users were asked how long they typically engage with the software whe they use it to do their work. Fifty percent of the users answered that they use the software between 15 to 30 minutes whilst 40\% of the users answered that they engaged with the software from 30 minutes to an hour at a time. Only one user answered that they used the software for fifteen minutes or less.


\subsubsection{Individual Impact}
Individual impact is more of a qualitative measure that is based on the users' personal interactions with the software and its effects that it has had on them. Individual impact includes the user's confidence when using the software, improved personal productivity, the amount of time it takes to complete a task as compared to when the user did not have the software, as well the amount of effort that is put into using the software. For this portion of the survey, the user was asked to estimate how much time they spent completing tasks before the software was implemented and how much time it took to complete the same task when using the software. The user was also asked if the software met their needs that were defined in their initial request for software; this question asked them to declare how much they agreed with the statement on a scale of one to five (five being that they completely agree and one being that they  completely disagreed). Finally, they were asked how much of their day-to-day responsibilities and work depended on their ability to use the newly developed application.


\begin{table}
\caption{Individual Impact - Survey Results. The scale is: Strongly agree (Str. Agr.), Agree (Agr.), Neutral (N.), Disagree (Dis.), and Strongly Disagree (Str. Dis.)}
\label{table:usersatisfaction}
\begin{tabular}{p{2.6cm}p{.75cm}p{.75cm}p{.75cm}p{.75cm}p{.75cm}}
Question: & Str. Agr. & Agr. & N. & Dis. & Str. Dis. \\
 \hline
The application meets my needs & 50\% & 40\% & 10\% & 0\% & 0\% \\
Completing a task before the software took a lot of time & 50\% & 20\% & 10\% & 20\% & 0\% \\
Completing a task after the software took less time than before & 70\% & 0\% & 10\% & 20\% & 0\% \\
\end{tabular}
\end{table}


\subsubsection{Organizational Impact}
Organizational impact evaluates the organizational goals of the institution, cost reduction, increase in the efficiency and the effectiveness of services. The improvement of the overall institution* directly correlates to the improvements in the users' day-to-day tasks. To gauge what impact the software has had on the institution as a whole, the users were asked to rate the impact, in regards to improvement, the solution that the software provided on a scale of one to four (four being significant improvement, one being no improvement at all). The users were also asked what it would cost in terms of money, stress, errors, and time if the new software was suddenly completely taken down and they had to revert back to their previous processes. This question also required the users to evaluate this on a scale of one to five (five being that it would cost them very much, one not costing them anything at all.) Last, the users were asked to estimate the value of the software in comparison to other applications used at the institution that were not built by the SSDT. They were also to evaluate this comparison on a scale of one to five (five being more valuable than the other software, one being less valuable than the other software).

\begin{table}
\caption{Organizational Impact - Survey Results. The scale is: Strongly agree (Str. Agr.), Agree (Agr.), Neutral (N.), Disagree (Dis.), and Strongly Disagree (Str. Dis.)}
\label{table:usersatisfaction}
\begin{tabular}{p{2.8cm}p{.7cm}p{.7cm}p{.7cm}p{.7cm}p{.7cm}}
Question: & Str. Agr. & Agr. & N. & Dis. & Str. Dis. \\
 \hline
My day-to-day tasks \& work depend on software & 60\% & 40\% & 0\% & 0\% & 0\% \\
Software improved the institution as whole. & 70\% & 20\% & 0\% & 10\% & 0\% \\
If removed, it would cost me a lot of time. & 80\% & 10\% & 10\% & 0\% & 0\% \\
If removed, it would cost me a lot of stress. & 40\% & 40\% & 20\% & 0\% & 0\% \\
If removed, it would cost me a lot of money. & 50\% & 10\% & 10\% & 0\% & 30\% \\
If removed, it would cost me a lot of errors. & 40\% & 40\% & 20\% & 0\% & 0\% \\
\end{tabular}
\end{table}
