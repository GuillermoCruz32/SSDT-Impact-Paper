\section{RESULTS}

\subsection{Measuring Success}

The objective of the study is to demonstrate the impact of software developed predominantly by undergraduate students has on higher educational institutions. To determine the impact that the student-developed software has had on the institution several factors of software success are identified and evaluated.  There exists much research on quantifying project success. For this study, the metrics for software success were defined using DeLone and McLean’s Model for Information Software Success. (CITATION NEEDED) This model consists of six measures of success which included the following: system quality, information quality, use, user satisfaction, individual impact, and organizational impact. These measures may be used to evaluate the software systems based on factors including but not limited to, code quality, appearance, usability, relevance, and cost reduction. The software developed by the SSDT uses ideas outlined in the Agile Manifesto. A survey comprised of questions that derive from each of these six components as well as principles from the Manifesto was sent out to Product Owners of software that was built by the SSDT.

As previously stated, DeLone's and McLean's Model for Software Success were used to model the questions within the survey. Particularly, the last four measures were focused on for the survey as the first two (system quality and information quality) are two factors that the Product Owners will not have much interaction with and is more technical, so they were not included in the survey. These four factors* are user satisfaction, use, individual impact,and organizational impact.

\subsubsection{User satisfaction}
User satisfaction takes into consideration the ease of usefulness of the software. The questions in the survey asked about the simplicity of completing a specific task pertaining to the individual software that Product Owner would have interacted with as well as how learnable the software is. The Product Owners were also asked to rate how useful the features in the software were to them on a scale of one to four (four being the highest).

\subsubsection{Use}
Use of a software factors in the frequency of use, the length of time using the software, and the number of users who actually utilize the software. For this component, the users were given four statements and were asked to rate each statement based on the degree of their concordance with the statement on a scale of one to five (five being that they completely agreed with the statement and one being that they did not agree with the statement at all.) The statements included the following: The application allows me to do my job, The application makes my job better, The application makes my job faster, I feel confident while using the application,  and last, My level of competency with software is not relevant when using the application. The survey also posed questions about how often each user engaged with the application (daily, weekly, monthly, yearly, once per semester, or never)  and how long the user interacts with the application (less than 15 minutes, between 15 and 30 minutes, between 30 minutes and an hour, between one to two hours, and over two hours per session)

\subsubsection{Individual Impact}
Individual impact is more of a qualitative measure that is based on the users personal interactions with the software and its effects that it has had on them. Individual impact includes the user's confidence when using the software, improved personal productivity, the amount of time it takes to complete a task as compared to when the user did not have the software, as well the amount of effort that is put into using the software. For this portion of the survey, the user was asked to estimate how much time they spent completing tasks before the software was implemented and how much time it took to complete the same task when using the software. The user was also asked if the software met their needs that were defined in their initial request for software; this question was asked them to what degree that they agreesd with this statement on a scale of one to five (five being that they completely agree and one being that they completely disagreed).

\subsubsection{Organizational Impact}
Organizational impact evaluates the organizational goals of the institution, cost reduction, increase in the efficiency and the effectiveness of services. The improvement of the overall institution* directly correlates to the improvements in the users' day-to-day tasks. To gauge what impact the software has had on the institution as a whole, the users were asked to rate the impact, in regards to improvement, of the solution that the software provided on a scale of one to four (four being significant improvement, one being no improvement at all). The users were also asked what it would cost in terms of money, stress, errors, and time if the new software was suddenly completely taken down and they had to revert back to their previous processes. This question also required the users to evaluate this on a scale of one to five (five being that it would cost them very much, one not costing them anything at all.) Last, the users were asked to estimate the value of the software in comparison to other applications used at the institution that were not built by the SSDT. They were also to evaluate this comparison on a scale of one to five (five being more valuable than the other software, one being less valuable than the other software).


\subsection{Evaluating the Software}
 The first measure of success is the system quality which ca be defined by the code quality and quality of the system's documentation. The code quality can be measured by its conciseness, testabilty, usability, efficiency, structure, and clarity. These components are


\subsection{Participants}
Given the small size of our nonprofit institution, a limited number of staff have administrative access. Participants in this study included [INSERT NUMBER of PEOPLE HERE] Berea College faculty and staff. All participants in this study were volunteers and were administered a survey pertaining only to the software that they had interacted with. The survey was only sent to those who were the primary users and/or had administrative privileges to the software. This pool of faculty members was selected, as opposed to surveying all users, because these subjects were involved in the initial phase of the software development process, which is defining the software project’s scope and the needs of the customer. [CITATION?] Each survey began with an informed consent disclaimer that detailed the nature of the survey.

The SSDT has eight systems that we handle. These include the Advancement Office, BCAC, BCSR, CAS, LSF, Ulmann, URCPP, and SKYZ. Each system has a specific functionality to aid our faculty and staff clientele perform their jobs. For example, CAS (Course Administration and Scheduling) helps the registrar, as well as faculty and staff, schedule terms, manage courses, programs, divisions, and much more.
We distributed our surveys and gave our clients two weeks to submit their responses. We got just over 50 percent participation with eleven out of seventeen clients answering.

\subsection{Quantitative data (stats)}

%I think we should include all possible responses for questions like "how important is it" to show what responses the clients were given to choose from
%Yo I now think the idea above is trash. i trust you to translate data into these nice pretty academic terms. -Kat
[BRIA HALLLLLLLLLLLPPPPP MEEEEEE PLSSSSSSSSSS]
Of our 9 responses: %I started at column D in Data Analysis spreadsheet since the first three were not quantitative -kat
How often do you use the software? daily use: 3, weekly use: 1, monthly use: 4, yearly use: 1.
Each time you use the  application, how long do you interact with it?: < 15 minutes: 1, 15-30 mins: 3, 30 mins - 1 hr: 4, 1 - 2 hrs: 1.
How much of your day-to-day responsibilities depend on your ability to use the  application? Important: 4, Essential: 5.
How much do you agree with each of the following statements?
Using this application allows me to do my job better: Strongly Disagree: 1, Neutral: 1,Agree: 1, Strongly Agree: 6.
Using this application allows me to do my job faster: Strongly Disagree: 1, Disagree: 1, Agree: 1, Strongly Agree: 6
My level of competency with software is not relevant when using the software Disagree: 1, Neutral: 4, Agree: 2 Strongly Agree: 2
%The next questions are system specific functionality/ease of use so we will have to represent that data in some other way...Since each system had different (sometimes 1, sometimes multiple, functionalities here (cas has a lot, BCSR has 1.)
%If system was gone, what would it cost client?
If the software were to be taken down entirely, what would it cost you in terms of:
Stress:
Time:
Errors:
Money:
%Comparison of system to other non SSDT software(moodle,banner, etc)
How would you estimate the value of the software compared to Berea College's following applications?
Banner:
MyBerea:
Degree Works:
Moodle:
Online Bookstore:
TRACY:
College email:
Password Management site:
Hutchins library site:
